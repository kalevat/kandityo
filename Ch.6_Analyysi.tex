\chapter{Analyysi\label{analyysi}}
Tutkimuksessa verrattiin konesalipalvelun ja pilvipalvelun laatua raportoitujen tikettien perusteella sekä IT henkilön haastatteluiden pohjalta. Tutkimuksen tuloksista laadittiin analyysi, jossa vastataan tutkimuskysymykseen, kumpi palvelumalleista on laadultaan parempi. Tutkimuksen tavoitteena oli myös selvittää tekikö Finnair strategisesti oikean ratkaisun siirtyessään täysimääräisesti pilvipalveluiden käyttäjäksi.

Tikettitilastoja verrattaessa huomataan, että konesalipalveluissa on ollut enemmän kriittisiä tikettejä kuin pilvipalveluissa.  Pilvipalveluissa vuoden 2021 aikana kriittisiä tikettejä on kuukausittain ollut vain muutamia kappaleita. Verrattaessa virhetilanteista aukaistujen tikettien kokonaismäärää huomataan, että tikettejä on lähes yhtä paljon kuukausittain pilvipalveluissa ja konesalipalveluissa. Konesalipalveluista ei ollut tutkimukseen saatavissa automaattitikettien tilastoja, joten analyysiä automaattitikettien osalta ei voida tutkimuksessa tehdä. Tuloksien perusteella voidaan todeta, että tikettimäärä on molemmissa palvelumalleissa ollut lähes sama, mutta pilvipalveluissa kriittisiä virheitä on ollut vähemmän. Tutkimuksen analyysinä voidaan todeta, että pilvipalveluiden laatu on ollut parempi tarkastellulla aikajaksolla.

Tutkimuksessa verrattiin palvelumalleja myös palvelupyyntötikettien perusteella. Tuloksista voidaan todeta, että palvelupyyntöjä on tehty yhtä paljon konesalipalveluissa ja pilvipalveluissa. Pilvipalveluissa palvelupyyntöjen ratkaisuaika on ollut keskimäärin reilu 10 päivää ja konesalipalveluissa keskimääräin 2,5 päivää. Konesalipalveluissa palvelupyynnöt on ratkaisu nopeammin. Konesalipalveluissa maksimiarvo ratkaisuajalla on ollut alhaisempi kuin pilvipalveluissa. Tuloksien perusteella voidaan todeta, että konesalipalveluissa palvelupyyntöjen ratkaisunopeus on ollut nopeampi ja sen osalta laatu ollut parempaa käyttäjille.

Haastatteluissa IT asiantuntija ja IT johtaja toivat esiin erilaisia näkökulmia verrattaessa konepalveluja pilvipalveluja. Haastattelujen perusteella voidaan todeta, että tuotanto on pilvipalvelujen kautta ollut jonkin verran stabiilimpaa, mutta laadun ero pilvipalvelujen ja konesalinpalvelujen välillä ei ole ollut iso. Pilvipalveluissa on ollut vähemmän infrastruktuuriin liittyviä ongelmia kuin konesalipalveluissa. Pilvipalveluissa ongelmien sattuessa vaikutukset ovat olleet suuremmat ja laajemmat. Tosin pilvipalveluissa laajempia ongelmia on harvemmin sattunut. Haastatteluiden perusteella virhetilanteita on pilvipalveluissa ollut vähemmän ja virhetilanteet ovat olleet lyhytkestoisempia kuin aikaisemmin konesalipalveluiden kanssa. Ongelmatilanteet on ratkaistu pilvipalveluissa nopeammin.

Kun verrataan pilvipalveluja ja konesalipalveluja palvelupyyntöjen näkökulmasta, niin palveluissa ei ole ollut merkittävää eroa haastattelujen perusteella. Pilvipalvelut ovat ketteriä ja mahdollistavat nopean reagoinnin palvelupyyntöihin. Konesalipalveluissa tarvitaan enemmän työtä ja byrokratiaa pieniinkin muutoksiin. Pilvipalvelut ovat vakioituja automaattisia prosesseja, jotka mahdollistavat nopean toiminnan, mutta toisaalta pilvipalveluiden ylläpito on haastavampaa ja vaatii yhtenäisen toimintatavan asiakkaalta. Toisaalta pilvipalvelut ovat liiankin joustavia ja tietoturvaan tulee erityisesti kiinnittää huomiota.

Tutkimuksen haastatteluissa kysyttiin haastateltavilta yleisnäkemystä, siitä kumpi toimintamalli on parempi Finnairille, kun verrataan konesalipalveluja ja pilvipalveluja.  Pilvipalveluja pidettiin oikeana suuntana toiminnan parantamiseen. Pilvipalvelut pidettiin joustavampana ja sitä kautta taloudellisten säästöjen tekeminen on nopeampaa. Pilvipalvelut ovat Finnairin strategian mukainen valinta, koska pilvipalvelut antavat enemmän mahdollisuuksia kuin konesalipalvelu. Finnairin tulee myös laajemmin muuttua toimintatavoiltaan ketterämmäksi.

Tutkimuksen tuloksista voidaan todeta, että pilvipalvelut ovat oikea ratkaisu Finnairin kannalta. Pilvipalveluihin siirto on vielä kesken, siksi kaikkia hyötyjä ja haittoja ei ole vielä havaittu. Virhetilanteista raportoituen tikettien perusteella pilvipalveluissa laatu on ollut parempaa ja vakavia häiriötilanteita on ollut vähemmän. Konesalipalveluissa palvelupyyntöjen ratkaisunopeus oli nopeampi ja sen osalta laatu oli parempaa käyttäjille. Haastatteluissa pilvipalveluja pidettiin joustavampana ja asiantuntijat pitivät tärkeänä, että muutoksiin voidaan reagoida nopeasti. Haastatteluissa pilvipalveluille annettiin parempi kokonaisarvosana. 


