\chapter{Palvelun laatu\label{laatu}}
Palvelutoiminnassa tuotannon laatu on kriittinen tekijä. Palvelun tuottaja on velvollinen tarjoamaan sovittua laatutasoa palveluistaan. Palveluiden laatu on myös tarjouspyynnöissä ja kilpailutuksessa tärkeä valintatekijä. Konesalipalveluille ja pilvipalveluille on asetettu maailmanlaajuisten standardien kautta laatukriteerit, jotka niiden tulee täyttää. Pääsääntöisesti kaikki palvelutarjoavat täyttävät nämä ISO standardin mukaiset laatukriteerit. Laadun mittaamisella osoitetaan avoimesti asiakkaille kuukausittain ja vuosittainen laatutaso, johon on vertailujaksolla päästy. Nämä mittaamismenetelmät ovat vakiintuneet käytäntöjä IT-palvelutoimittajien parissa.
\section{Laatuvaatimukset}
\section{Laadun mittaaminen}
Ohjelmistotuotannon palvelun laadun mittaaminen pohjautuu ISO 25000 standardiin, joka tunnetaan myös nimellä (System and Software Quality Requirements and Evaluation) SQuaRE. Standardi asettaa kehityksen ohjelmistotuotannon laadun mittaamiseen \citep{iso25000}.  Standardin ISO 25010 mukaisesti hyvä tapa on jakaa tuotteen laatu eri ominaisuuksiin \citep{iso25010} Näiden standardien pohjalta Quality Model of Cloud Service julkaisussa ehdotetaan pilvipalveluiden mittaamista kuudella kriteerillä: käytettävyys (\emph{usability}), turvallisuus (\emph{security}), luotettavuus (\emph{reliability}), konkreettisuus (\emph{tangibility}), reagointikyky (\emph{responsiveness}) ja empatia (\emph{empathy}) \citep{qualitymodel}.

Käytettävyys (\emph{usability}) tarkoittaa pilvipalveluissa, että asiakkaalla on mahdollisuus itsenäisesti tilata ja käyttää palveluja. Tärkeintä asiakkaalle on, että pilvipalvelun käyttö on helppoa, tehokasta ja nautittavaa \citep{adaptive}. Nämä kriteerit saadaan täytettyä hyvillä ohjeilla, helppokäyttöisellä käyttöliittymällä ja selkokielisillä virheilmoituksilla. Näin saadaan asiakkaalle subjektiivinen tuntemus käytettävyydeltään hyvästä pilvipalvelusta \citep{qualitymodel}.

Turvallisuus (\emph{security}) on tärkeä palvelukriteeri kaikille pilvipalveluille. Keskitettyjen palveluiden haasteena on rakentaa riittävän turvallinen palvelu asiakkaille. Pilvipalveluiden tulee tarjota asiakkaille toimintaympäristö, jossa tietojen luottamuksellisuus ja eheys sekä asiakkaiden yksityisyyden suoja on tarkkaan huomioitu. Pilvipalvelujen tarjoajan tulee myös huolehtia jäljitettävyydestä, jossa palveluprosessin toiminnot tallennetaan ja ne tulee voida jäljittää jälkikäteen \citep{qualitymodel}.

Luotettavuus (\emph{reliability}) on olennainen ominaisuus pilvipalvelupohjaiselle IT-palvelulle. Asiakkaalle tärkeitä ominaisuuksia palvelulle on saatavuus ja jatkuvuus. Asiakkaan odottavat, ettei pilvipalvelussa ole laitteisto-ongelmia, ohjelmistovikoja tai muita vikoja, jotka voidaan aiheuttaa ongelmia palvelun saatavuuteen. Luotettavuuden kriteerejä ovat saatavuus (availability), täydellisyys (completeness), oikeudenmukaisuus (correctness), jatkuvuus (continuity) ja vakaus (stability). Asiakkaiden vaatimuksena on, että palvelu on ympärivuorokautisesti saavutettavissa ilman katkoja \citep{qualitymodel}.

Konkreettisuus (\emph{tangibility}) on pilvipalvelun tarjoajan mittari avoimuudelle. Palveluntarjoajan tulee antaa asiakkaalle näkyvyyttä palvelun laatutasoon raporttien muodossa. Lisäksi palveluntarjoajan tulee toimia ammattimaisesti niin, että kehitystoimenpiteet perustuvat asianmukaisiin taitoihin ja asiantuntemukseen.  Tämä käyttäjäkokemus on mahdollista tarjota asiakkaalle hyvän käyttöliittymän kautta \citep{qualitymodel}.

Reagointikyky (\emph{responsiveness}) tarkoittaa tietojenkäsittelyssä järjestelmän tai toiminnallisen yksikön kykyä suorittaa sille osoitetut tehtävät tietyssä ajassa \citep{dictionary}. Pilvipalvelutarjoajan tulee olla valmis ja kykenevä auttamaan asiakkaita tarjoamalla nopeaa ja oikea-aikaista palvelua. Reagointikyvyn ominaisuuksia ovat ajantasaisuus, vuorovaikutus sekä kyky toimittaa resursseja asiakkaan tarpeiden mukaisesti \citep{qualitymodel}.

Empatia (\emph{empathy}) tarkoitetaan, että pilvipalvelun tarjoaja on kykenevä muokkaamaan palveluaan yksittäisten asiakkaiden tarpeiden mukaisesti. Asiakaskokemus syntyy kokemuksien mukaan. Pilvipalveluntarjoajan tulisi kattaa asiakkaan vaatimukset ja tarjoat asiakkaalle laadukasta palvelua. Palvelutarjoaja voi saavuttaa nämä ominaisuudet panostamalla itsepalveluun, palveluiden mitattavuuteen ja aloitekykyyn. Palveluntarjoajan tulisi olla proaktiivinen ja tunnistaa asiakkaan vaatimukset sekä ehdottaa muutoksia tarpeiden mukaisesti \citep{qualitymodel}.
