\begin{abstract}

Tutkimuksessa verrattiin tuotannon laatua pilvipalveluissa ja konesalipalveluissa. Tutkimuksessa selvitettiin näiden  palveluiden eroja ja palvelutarjoajia. Tutkimuskohteena oli Finnair Oyj IT-järjestelmät, jotka siirtyivät pilvipohjaiseen järjestelmään vuoden 2021 alusta. Sitä ennen palvelut tuotettiin konesalipalveluina.

Tutkimuksessa tarkasteltiin kirjattujen tikettien perusteella erikseen virhetilanteita ja palvelupyyntöjä konesalipalveluissa ja pilvipalveluissa. Virhetilanteissa tarkastelussa oli tikettien tärkeysaste ja erityisesti kriittiset virhetilanteet. Palvelupyyntöjen osalta tarkasteltiin ratkaisuaikoja, jotka vaikuttavat palveluiden laatukokemukseen. Tutkimuksessa haastateltiin kahta Finnairin IT työntekijää, jotka tuntevat konesalipalvelut ja pilvipalvelut. Haastatteluissa kerättiin tietoa palveluiden laadusta subjektiivisena kokemuksena. Tutkimuksen ulkopuolelle jätettiin pilvipalveluiden ja konesalipalveluiden kustannukset, koska nämä tiedot eivät ole julkisia.

Tutkimuksen havainnoista laadittiin kirjallinen analyysi. Tutkimuksen tuloksena oli, että pilvipalvelut ovat strategisesti oikea ratkaisu Finnairin kannalta. Pilvipalveluiden laatu on ollut parempaa, kun verrataan virhetilanteista raportoitujen tikettien määrää.  Pilvipalveluissa vakavia häiriötilanteita on ollut vähemmän. Konesalipalveluissa palvelupyyntöjen ratkaisunopeus oli nopeampi ja palvelukokemus oli parempi käyttäjille. Haastatteluissa pilvipalveluja pidettiin joustavampana, joka mahdollista nopeamman reagoinnin muutoksiin. Kokonaisarvosana haastatteluissa pilvipalveluja pidettiin parempana vaihtoehtona.

\end{abstract}
