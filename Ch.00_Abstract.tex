\begin{abstract}

Tutkimuksessa verrattiin tuotannon laatua pilvipalveluissa ja konesalipalveluissa. Tutkimuksessa selvitettiin näiden eri palveluiden eroja ja palvelutarjoajia. Tutkimuskohteena oli Finnair Oyj IT-järjestelmät, jotka siirtyivät pilvipohjaiseen järjestelmään vuoden 2021 alusta. Sitä ennen palvelut tuotettiin konesalipalveluina.

Tutkimuksessa tarkasteltiin kirjattujen tikettien perusteella erikseen virhetilanteita ja palvelupyyntöjä konesalipalveluissa ja pilvipalveluissa. Virhetilanteissa tarkastelussa oli tikettien tärkeysaste ja erityisesti kriittiset virhetilanteet. Palvelupyyntöjen osalta tarkasteltiin ratkaisuaikoja, jotka vaikuttavat palveluiden laatukokemukseen. Tutkimuksessa haastateltiin kahta Finnairin IT työntekijää, jotka tuntevat konesalipalvelut ja pilvipalvelut. Haastatteluissa kerättiin tietoa palveluiden laadusta subjektiivisena kokemuksena. Tutkimuksen ulkopuolelle jätettiin pilvipalveluiden ja konesalipalveluiden kustannukset, koska nämä tiedot eivät ole julkisia.

Tutkimuksen tuloksista laadittiin analyysi havainnoista. Tutkimuksen tuloksena oli, että...



\end{abstract}
