\begin{abstract}

Tutkimuksessa verrattiin tuotannon laatua pilvipalveluissa ja konesalipalveluissa. Tutkimuksessa selvitettiin näiden eri palveluiden eroja ja palvelutarjoajia. Tutkimuskohteen aoli Finnair Oyj IT-järjestelmät, jotka siirtyivät pilvipohjaiseen järjestelmään vuoden 2021 alusta. Sitä ennen palvelut tuotettiin konesalipalveluista.

Tutkimuksessa tarkasteltiin kirjattuen tikettien perusteella erikseen virhetilanteita ja palvelupyyntöjä konesalipalveluissa ja pilvipalveluissa. Virhetilanteissa tarkastelussa oli tikettien tärkeyaste ja erityisesti kriittiset virhetilanteet. Palvelupyyntöjen osalta tarkasteltiin ratkaisuaikoja, jotka vaikuttavat palveluiden laatukokemukseen. Tutkimuksessa haastateltiin kahta Finnairin työntekijää, jotka tuntevat konesalipalvelut ja pilvipalvelut. Haastteluissa kerättiin tietoa palveluiden laadusta subjektiivena kokemuksena.

Tutkimuksen tuloksista laadittiin analyysi havainnoista. Tutkimuksen tuloksena oli, että...



\end{abstract}
