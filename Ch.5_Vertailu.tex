\chapter{Palveluiden vertailu\label{vertailu}}

\section{Taustatiedot}
Pilvipalvelujen ja konesalipalveluiden laatua verrattiin tutkielmassa keräämällä aineisto, joka sisälsi raportoidut tiketit eri palvelutoimittajien järjestelmiin. Tutkielmassa tarkasteltiin erikseen palveluongelmia (\emph{incident}) ja palvelupyyntöjä (\emph{service request}). Molemmista palveluista oli saatavissa tutkielmaa varten kattavat tilastot tiketeistä. Lisäksi tutkielmaa varten haastateltiin Finnairin työntekijöitä ja pyydettiin heitä kokemuksia pilvi- ja konesalipalveluista. Näin saatiin myös subjektiivinen näkökulma tutkielmaan.

Isot, laaja-alaiset ja vakavat ongelmatilanteet luokitellaan termillä \emph{Major Incident}. Tässä tilanteessa virhetilanteesta lähetetään sähköpostitse tiedote yhtiön sisäisesti eri sidosryhmille ja ongelmatilanteen selvitykseen nimitetään oma asiantuntija. Nämä vakavat ongelmatilanteet on erikseen tilastoitu ja niistä tehdään myös jälkikäteen erillinen selvitys syistä. Vakavat ongelmat voivat olla haitaksi lentoliikenteelle ja sitä kautta myös Finnairin ydinliiketoiminnalle. Näitä vakavia virhetilanteita tulisi välttää kaikin keinoin ja niistä tulisi toipua mahdollisimman nopeasti. Tutkielmassa perehdyttiin näiden vakavien virheiden määrään ja vakavuusluokitukseen.
\section{Laatuvertailut}
Tutkielmassa verrattiin konesalipalvelun aikaisia tikettejä sekä pilvipalveista raportoituja tikettejä. Vertailua tehtiin erikseen virhetilanteista ja palvelupyynnöistä. Tutkielmassa tarkasteltiin tikettien määrää sekä tikettien tärkeysastetta (\emph{priority}). Palvelupyyntöjen osalta tarkasteltiin myös tikettien ratkaisuaikoja, koska se vaikuttaa palvelukokemukseen.

Tiketit on luokiteltu viisiportaisella asteikolla eri tärkeysasteille. Samaa luokittelua on käytetty konesalipalveluissa ja pilvipalveluissa, joten tilastot ovat vertailukelpoisia.  Luokittelu perustuu virhetilanteen kriittisyyteen ja virhetilanteen vaikutukseen tuotantoon. P1 tason tiketin aukaistaan tilanteessa, jossa ohjelmalla on kriittinen yrityksen tuotannolle ja kyseisellä virhetilanteella on iso laaja-alainen vaikutus. Osa virhetilanteista on automaattitikettejä, jotka valvontalaitteisto aukaisee automaattisesti. Muut tiketit ovat käsin aukaistuja tikettejä niin, että loppukäyttäjä raportoi ongelmasta IT toimittajan asiakaspalveluun. 

Pilvipalveluissa tarkasteltava aikajakso on 1.1.--30.9.2021. Tänä aikana virhetilanneisiin liittyviä käsin avattuja kriittisiä tikettejä on ollut yhteensä 22 ja vähemmän kriittisiä tikettejä 194. Kuvassa \ref{fig:pilvitiketit} nähdään tikettien määrät vuoden 2021 eri kuukausina sekä jakautuma kriittisten ja vähemmän kriittisten tikettien välillä. Kuvasta voidaan todeta, että kriittisiä virhetilanteita on ollut vähäinen määrä kuukausittain. Pääasiassa virhetilanteet ovat olleet vähemmän kriittisiä. Kuvasta nähdään, että kriittisiä virhetilanteiden määrässä ei ole ollut kuukausittain isoa vaihtelua. Taulukosta \ref{table:pilviautomaatti} nähdään kuinka paljon eri tärkeysasteen tikettejä on ollut pilvipalveluissa vuoden 2021 aikana ja miten tiketit on jakautuneet käsin tehtyjen ja automaattitikettien välillä. Taulukosta voidaan todeta, että kriittisiä automaattitikettejä on ollut vähän verrattuna kokonaismäärään. Taulukosta nähdään, että käsin on avattu pääasiassa P3 tason tikettejä, jotka ovat vähemmän kriittisiä.

\begin{filecontents}{data1.dat}
X Time  	Part1  Part2
1 tammi  	2	    22
2 helmi		1	    25
3 maalis	2	    18
4 huhti		3	    41
5 touko		2	    28
6 kesä		2	    13
7 heinä		2	    14
8 elo       3       15
9 syys      5       18
\end{filecontents}

\begin{figure}[ht]
\begin{tikzpicture}
\begin{axis}[
axis lines=middle,
ymin=0,
legend pos=outer north east,nodes near coords,
x label style={at={(current axis.right of origin)},anchor=north, below=10mm},
title={\textbf{\textit{}}},
    xlabel=Vuosi 2021,
    ylabel=Määrä,
    xticklabel style = {rotate=30,anchor=east},
    enlargelimits = false,
    xticklabels from table={data1.dat}{Time},xtick=data]
\addplot[blue,thick,mark=square*] table [y=Part1,x=X]{data1.dat};
\addlegendentry{P1+P2 tiketit}
\addplot[red,thick,mark=square*] table [y= Part2,x=X]{data1.dat};
\addlegendentry{Muut tiketit}
\end{axis}
\end{tikzpicture}
\caption{Vertailu käsin avattujen kriittisten ja vähemmän kriittisten tikettien välillä pilvipalveluissa}
\label{fig:pilvitiketit}
\end{figure}

\begin{table}[ht]
\centering
\begin{tabular}{||c c c||} 
 \hline
 Kriittisyys & Käsin avatut tiketit & Automaattitiketit \\ [0.5ex] 
 \hline\hline
 P1 & 5 & 14 \\ 
 P2 & 19 & 0 \\
 P3 & 173 & 17 \\
 P4 & 9 & 1 \\
 P5 & 9 & 348 \\
 \textbf{Yhteensä} & \textbf{215} & \textbf{380}\\ [1ex] 
 \hline
\end{tabular}
\caption{Pilvipalveluissa avatut tiketit 1.1.-30.9.2021}
\label{table:pilviautomaatti}
\end{table}

Pilvipalveluissa oli aikajaksolla 1.1.--30.9.2021 yhteensä 224 palvelupyyntöä. Palvelupyyntöjen ratkaisuaika oli keskimäärin 244 tuntia eli reilu 10 päivää. Maksimiarvo ratkaisuajalla oli 2213 tuntia eli yli 92 päivää. Kuvassa \ref{fig:pilvipyynto} nähdään palvelupyyntötikettien määrät vuoden 2021 eri kuukausina. Kuvasta voidaan todeta, että palvelupyyntöjen määrä on ollut nousussa. Nousu johtuu siitä, että pilvipalvelujen käyttöönotto on edelleen kesken ja muutoksia ympäristöön tehdään jatkuvasti Lisäksi Finnairin ydinliiketoiminta on elpymässä ja liiketoiminnan kehitystä on alettu tekemään korona-ajan jälkeen.

\begin{filecontents}{data2.dat}
X Time  	Part1
1 tammi  	8
2 helmi		21
3 maalis	21
4 huhti		16
5 touko		38
6 kesä		34
7 heinä		23
8 elo       31
9 syys      32
\end{filecontents}

\begin{figure}[ht]
\begin{tikzpicture}
\begin{axis}[
axis lines=middle,
ymin=0,
legend pos=south east,nodes near coords,
x label style={at={(current axis.right of origin)},anchor=north, below=10mm},
title={\textbf{\textit{}}},
    xlabel=Vuosi 2021,
    ylabel=Määrä,
    xticklabel style = {rotate=30,anchor=east},
    enlargelimits = false,
    xticklabels from table={data2.dat}{Time},xtick=data]
\addplot[blue,thick,mark=square*] table [y=Part1,x=X]{data2.dat};
\addlegendentry{Palvelupyynnöt}
\end{axis}
\end{tikzpicture}
\caption{Tilasto palvelupyynnöistä pilvipalveluissa}
\label{fig:pilvipyynto}
\end{figure}

Konesalipalveluissa tarkasteltava aikajakso on 1.11.2019--30.11.2020. Aikajakso on valittu sen mukaan, että siltä ajalta oli parhaiten vertailukelpoista tietoa saatavissa tutkimukseen. Lisäksi konesalipalvelut olivat tuolla ajanjaksolla vakiintuneet ja tuotanto oli stabiilia. Ajanjaksona virhetilanneisiin liittyviä käsin avattuja kriittisiä tikettejä oli yhteensä 117 ja vähemmän kriittisiä tikettejä 156. Kuvassa \ref{fig:konesalitiketit} nähdään tikettien määrät vuoden 2021 eri kuukausina sekä jakautuma kriittisten ja vähemmän kriittisten tikettien välillä. Kuvasta voidaan todeta, että kriittisiä virhetilanteita on ollut vuoden 2019 lopussa enemmän kuin vuoden 2020 lopussa. Kuvasta voidaan huomata, että tikettien kokonaismäärä on laskenut vuoden 2020 aikana. Vuoden 2020 lopussa kriittisten tikettien määrä oli vähäinen.  Taulukosta \ref{table:koneautomaatti} nähdään kuinka paljon eri tärkeysasteen tikettejä on ollut konesalipalveluissa aikajaksolla 1.1.--30.9.2020 aikana. Taulukosta nähdään, että käsin on avattu eniten P3 tason tikettejä, jotka ovat vähemmän kriittisiä. Taulukon mukaan P1 ja P2 tikettejä 48 \% tikettien kokonaismäärästä. Taulukon lähdetiedoista puuttuu P5 tason tiketit.

\begin{filecontents}{data3.dat}
X Time  	Part1  Part2
1 marras  	22	    30
2 joulu		8	    20
3 tammi 	7	    15
4 helmi		19	    24
5 maalis	23	    16
6 huhti		6	    6
7 touko		4	    5
8 kesä      6       6
9 heinä     7       8
10 elo      3       8
11 syys     3       4
12 loka     7       11
13 marras   2       3
\end{filecontents}

\begin{figure}[ht]
\begin{tikzpicture}
\begin{axis}[
axis lines=middle,
ymin=0,
legend pos=north east,nodes near coords,
x label style={at={(current axis.right of origin)},anchor=north, below=10mm},
title={\textbf{\textit{}}},
    xlabel=Vuosi 2019-2020,
    ylabel=Määrä,
    xticklabel style = {rotate=30,anchor=east},
    enlargelimits = false,
    xticklabels from table={data3.dat}{Time},xtick=data]
\addplot[blue,thick,mark=square*] table [y=Part1,x=X]{data3.dat};
\addlegendentry{P1+P2 tiketit}
\addplot[red,thick,mark=square*] table [y= Part2,x=X]{data3.dat};
\addlegendentry{Muut tiketit}
\end{axis}
\end{tikzpicture}
\caption{Vertailu käsin avattujen kriittisten ja vähemmän kriittisten tikettien välillä konesalipalveluissa}
\label{fig:konesalitiketit}
\end{figure}

\begin{table}[ht]
\centering
\begin{tabular}{||c c||} 
 \hline
 Kriittisyys & Käsin avatut tiketit \\ [0.5ex] 
 \hline\hline
 P1 & 37 \\ 
 P2 & 48 \\
 P3 & 85 \\
 P4 & 14 \\
 P5 &  \\
 \textbf{Yhteensä} & \textbf{184} \\ [1ex] 
 \hline
\end{tabular}
\caption{Konesalipalveluissa käsin avatut tiketit vuonna 1.1.--30.9.2020}
\label{table:koneautomaatti}
\end{table}

Konesalipalveluissa oli aikajaksolla 1.11.2019--30.11.2020 yhteensä 3374 palvelupyyntöä. Palvelupyyntöjen ratkaisuaika oli keskimäärin 244 tuntia eli reilu 10 päivää. Maksimiarvo ratkaisuajalla oli 2213 tuntia eli yli 92 päivää. Kuvassa \ref{fig:konepyynto} nähdään palvelupyyntötikettien määrät eri kuukausina. Kuvasta voidaan todeta, että palvelupyyntöjen määrä on laskenut. Tämä selittyy sillä, että vuoden 2020 lopussa konesaliympäristöstä oltiin siirtämässä pilvipalveluihin eikä ympäristöä enää kehitetty.

\begin{filecontents}{data4.dat}
X Time  	Part1
1 marras  	497
2 joulu		388
3 tammi 	460
4 helmi		490
5 maalis	346
6 huhti		178
7 touko		52
8 kesä      114
9 heinä     156
10 elo      142
11 syys     214
12 loka     246
13 marras   91
\end{filecontents}

\begin{figure}[ht]
\begin{tikzpicture}
\begin{axis}[
axis lines=middle,
ymin=0,
legend pos=north east,nodes near coords,
x label style={at={(current axis.right of origin)},anchor=north, below=10mm},
y label style={at={(axis description cs:-0.2,.5)},rotate=90,anchor=south},
title={\textbf{\textit{}}},
    xlabel=Vuosi 2019-2020,
    ylabel=Määrä,
    xticklabel style = {rotate=30,anchor=east},
    enlargelimits = false,
    xticklabels from table={data4.dat}{Time},xtick=data]
\addplot[blue,thick,mark=square*] table [y=Part1,x=X]{data4.dat};
\addlegendentry{Palvelupyynnöt}
\end{axis}
\end{tikzpicture}
\caption{Tilasto palvelupyynnöistä konsesalipalveluissa}
\label{fig:konepyynto}
\end{figure}

\section{Haastattelut}
Haastatteluissa selvitettiin minkälainen subjektiivinen käsitys Finnair IT asiantuntijalla ja johtajalla on tuotannon laadusta, kun verrataan pilvipalvelua ja konesalipalvelua. Haastatteluun osallistuneilla henkilöillä on kokemusta molemmista toimintamalleista. Haastatteluissa haettiin heidän henkilökohtaisia mielipiteitä aiheesta ja mielipiteiden ei tarvinnut perustua todellisiin faktatietoihin. Faktatietoihin perustuva vertailua tehtiin tämän tutkimuksen muissa osissa. Subjektiivinen näkökulma on tärkeä selvittää, koska laatu perustuu myös tuntemukseen siitä kuinka hyvin asiakasta palvellaan sekä minkälainen olo hänelle on toiminnasta. Haastattelu jakautui kahteen osaan. Ensimmäisessä osassa selvitettiin tuotannon laatua virhetilanteiden (\emph{incident}) kautta ja toisessa osassa palvelupyyntöjen (\emph{service request}) kautta.

Haastattelussa todettiin, että tuotanto on pilvipalvelujen kautta ollut jonkin verran stabiilimpaa. IT asiantuntija korosti, että laadun ero pilvipalvelujen ja konesalinpalvelujen välillä ei ole ollut iso, mutta eron on silti huomannut. Pilvipalveluissa on ollut vähemmän infrastruktuuriin liittyviä ongelmia, kuten verkkoliikenteen ongelmat. Silti pilvipalveluissa on havaittu myös perinteisiä palvelimiin liittyviä ongelmia, kuten muistitilan ja levytilan loppuminen. Pilvipalveluissa ongelmien sattuessa vaikutukset ovat olleet suuremmat ja laajemmat. Tosin pilvipalveluissa näitä laajempia ongelmia on harvemmin sattunut, kun tilannetta verrataan konesalipalveluihin.

IT johtajan kiinnitti huomiota, että nykyinen pilviympäristö on vasta alle vuoden vanha ja kaikkia kokemuksia ei ole vielä saatu. Virhetilanteita on hänen mukaansa ollut vähemmän ja ne on ollut lyhytkestoisempia kuin aikaisemmin konesalipalveluiden kanssa. Nykyisessä AWS pilvipalvelussa tuotanto on hajautettu useammalle käyttöalueelle (\emph{AWS Zone}). Tällöin yhden käyttöalueen ongelmat eivät heijastu koko Finnairin tuotantoon. Tämä on auttanut siihen, että virhetilanteita on ollut vähemmän kuin konesalipalvelussa, jossa tuotanto toteutettiin yhdestä fyysisestä paikasta. Yhdessä konesalissa tapahtuva virhe heijastui laajasti koko tuotantoon. IT johtaja painotti, että tilanne tulee entisestään paranemaan pilvipalveluiden osalta, koska pilvipalveluiden toimittaja ja asiakkaan tulee molempien kehittää jatkuvasti ympäristöä eikä korjausvelkaa synny niin paljon kuin konesalipalveluissa.

Vakavia virhetilanteita asiantuntijan mukaan ollut nyt vähemmän, mutta jos verrataan tilannetta pidemmällä aikajaksolla, niin vakavia ongelmia on ollut suunnilleen sama määrä. Vakavien virhetilanteiden määrä on asiantuntijan mukaan riippuvainen kuinka paljon tehdään muutoksia ympäristöön. Jos tehdä isoja laajoja muutoksia IT ympäristöön, niin se sisältää aina riskejä, jotka voivat realisoitua vakavina ongelmina ja toimintahäiriöinä. Yrityksen pilvipalveluympäristö on sen verran uusi, ettei vielä ole tarvinnut tehdä isompia muutoksia ympäristöön, jolloin myös vakavia ongelmia ei ole syntynyt. Konesalipalveluissa on riski, että yksittäisen työntekijän tekemä inhimillinen virhe voi aiheuttaa isoja ongelmia. IT johtajan mukaan vakavia virhetilanteita on ollut pilvipalveluissa vähemmän eivätkä ongelmat ole olleet niin pitkäkestoisia. Tämä on näkynyt parempana tuotannon laatuna pilvipalveluissa ja parantanut Finnair IT palvelujen käytettävyyttä.

Asiantuntija mukaan ongelmien ratkaisunopeudessa ei ole ollut suurta ero kun verrataan pilvipalveluja ja konesalipalveluja. Konesalipalveluissa koko ympäristön hallinta oli keskitetty yhdelle yritykselle, jonka vastuulla oli kokonaisvaltaisesti pitää huolta, että palvelut toimivat. Ongelma tilanteiden selvitys oli tässä tilanteessa suoraviivaisempaa, koska vastuu yrityksiä oli vain yksi. Isolla konesalipalvelun tarjoajalla oli myös iso määrä eri alan asiantuntijoita valmiina selvittämään ongelmatilanteita. Tämä vaikutti ratkaisunopeuteen, kun oikeat asiantuntijat saatiin välittömästi mukaan selvitystyöhön. Pilvipalveluissa vastuut on hajautettu eri toimijoille sekä osa vastuista on otettu Finnairin hoidettavaksi. Tämä vaikuttaa ongelmien ratkaisunopeuteen, kun vastuualueet ovat epäselviä ja useampi toimittaja yrittää ratkaista ongelmaa. Finnairin asiantuntijan mukaan pilvipalveluissa on ongelmien ratkaiseminen ollut hitaampaa näistä syistä. Toisaalta yksinkertaisemmat ja pienemmät ongelmat ratkaistaan pilvipalveluissa nopeammin, kun ympäristö on topologialtaan selkeämpi.

IT johtaja oli samaa mieltä, että ongelmatilanteiden ratkaiseminen on ollut pilvipalveluissa nopeampaa kuin konesalipalveluissa. Pitää kuitenkin muistaa, että myös pilvipalveluissa voidaan tehdä isoja virheitä. Näistä ongelmista on kuitenkin pilvipalveluissa parempi näkyvyys, koska tiedot ovat julkisia. Konesalipalvelut ovat suljettuja ja niissä esiintyvät ongelmat toimittajan on mahdollista peittää. Konesalipalveluiden toimittaja voi kertoa vain niistä ongelmista, jotka ovat näkyneet asiakkaille. Pilvipalveluissa on ulkoisia tilannesivuja Internetissä, joissa näkyy ongelmatilanteet ja niihin tehdyt ratkaisut.

Kun verrataan pilvipalveluja ja konesalipalveluja palvelupyyntöjen (\emph{service request}) näkökulmasta, niin Finnairin asiantuntijan mukaan palveluissa ei ollut merkittävää eroa. Uuden pilvipalvelutoimittaan palvelualttius on ollut parempaa ja reagointi nopeampaa, mutta tähän voi vaikuttaa myös että uusi toimittaja haluaa näyttää asiakkaalle alkuun hyvää ja huolellista palvelua. Konesalipalvelut, jotka olivat keskitettyjä, toiminta ei ollut niin ketterää. Pilvipalvelut toimitetaan enemmän ketterien menetelmien mukaisesti, jolloin toiminta on joustavampaa. Konesalipalveluissa yksittäisen palvelupyynnön tekemiseen kului enemmän aikaa, koska jossain tapauksissa muutos haluttiin projektoida ja tämä loi lisää hidastavaa byrokratiaa. Pilvipalveluissa esimerkiksi uusien palvelimien lisääminen ympäristöön on hyvin suoraviivaista ja nopeaa. Näissä tilanteissa palvelupyyntöjen toteutus on nopeaa. Konesalipalveluissa palvelimien poisto palvelusta ei ole suoraviivaista. Palvelimesta ei voida pelkästään katkaista sähkövirta, vaan sille tulee tehdä myös muita poistotoimenpiteitä, joista tulee ylimääräisiä kustannuksia. Pilvipalveluissa taas palvelimien poisto on yksinkertaista ja nopeaa. Poisto voidaan tehdä konekielisesti yhdellä komennolla ja siitä syntyvät taloudelliset säästöt ovat välittömästi saatavissa.

IT johtajan mukaan pilvipalvelut ovat ketteriä ja nopean reagoinnin palvelupyyntöihin. Konesalipalveluissa tarvitaan enemmän työtä ja byrokratiaa pieniinkin muutoksiin. Pilvipalvelut ovat vakioituja automaattisia prosesseja, jotka mahdollistavat nopean toiminnan. Automatisoidut prosessit antavat myös enemmän ketteryyttä ja ovat varmempia laadultaan. Inhimillisten virheiden määrä on pienempi, kun prosessit hoidetaan automatisoidusti. Tämä antaa myös enemmän mahdollisuuksia ja joustoa asiakkaalle. IT johtaja piti tärkeänä, että pilvipalveluissa kiinnitetään huomiota tietoturvaan. Pilvipalvelut ovat liiankin joustavia, jolloin tietoturva unohtuu helposti, jollei siihen erityisesti kiinnitetä huomiota. Kaikki tietoturvaratkaisu eivät ole automatisoituja. Kaiken kaikkiaan pilvipalveluiden ylläpito on hänen mukaan haastavampaa ja vaatii yhtenäisen toimintatavan asiakkaalla. Pilvipalveluiden toimittaja ei pakota asiakasta tiettyihin toimintatapoihin. Myös IT arkkitehtuurin näkökulmasta pilvipalvelut korruptoituvat helpommin verrattuna konesalipalveluihin. IT johtaja tarkoitti korruptoitumisella sitä, että ympäristön monimutkaisuus tulee jossain vaiheessa vastaan, kun tehdään jatkuvaa kehitystä. Jossain vaiheessa pilviympäristö tulee muokata uudelleen, jollei arkkitehtuuri asioista pidetä riittävästi huolta muutostilanteissa.  

Tutkielman haastattelussa kysyttiin myös haastateltavilta yleisnäkemystä, siitä kumpi toimintamalli on parempi Finnairille, kun verrataan konesalipalveluja ja pilvipalveluja. Asiantuntija piti pilvipalveluja oikeana suuntana toiminnan parantamiseen. Asiantuntija painotti, että pilvipalvelut ovat huomattavasti joustavammat ja sitä kautta taloudellisten säästöjen tekeminen on nopeampaa sekä helpompaa. Pilvipalveluissa toimintaympäristö voidaan muokata itsenäisesti ilman isompaa byrokratiaa. Pilvipalveluihin siirtyminen oli väistämätön aihe ja se olisi tullut tehdä jossain vaiheessa. Asiantuntijan mukaan nyt oli hyvä aika tehdä muutostyö, kun koronaviruksen takia Finnairin toiminta oli hiljaisempaa. Finnair asiantuntija painotti, että pilvipalveluihin siirto on vielä kesken, siksi kaikkia hyötyjä ja haittoja ei ole vielä koettu. Tulevaisuus ei ole hänen mukaan niin selvä ja pilvipalvelujen toimintamallissa voi tulla myös yllätyksiä tulevaisuudessa. Kokonaisarvosanan (asteikolla 0-5) asiantuntija antoi konesaliyrityksen X palvelusta arvosanan 4- ja AWS pilvipalveluille arvosanan 4.

IT johtaja piti haastattelussa pilvipalveluja oikeana ratkaisuna Finnairin kannalta. Yrityksen tulee myös laajemmin muuttua toimintatavoiltaan ketterämmäksi. Tilanteet muuttuvat nykymaailmassa nopeasti ja muutoksiin tulee reagoida välittömästi. Pilvipalvelut ovatkin yrityksen strategian mukainen valinta. Pilvipalvelut antavat enemmän mahdollisuuksia kuin konesalipalvelut. IT johtaja muistutti, että pilvipalvelut vaativat enemmän teknistä osaamista myös asiakkaalta eikä palveluja voida enää ostaa niin helposti kuin konesalipalveluissa. Yrityksellä X oli tarjota konesalipalveluiden yhdessä lukuisia IT alan asiantuntijoita auttamaan ratkaisujen rakentamisessa. Tätä mahdollisuutta ei enää ole pilvipalveluiden kanssa, vaan asiakkaan tulee olla myös itse aktiivinen kun muutoksia tehdään. Pilvipalveluja tuleekin kehittää fiksusti huomioiden arkkitehtuuriset ratkaisut ja tietoturva. Kokonaisarvosanaksi (asteikolla 0-5) IT johtaja antoi konesaliyritykselle X palveluista 3 ja AWS pilvipalvelulle 4. IT johtaja painotti, että AWS arvosanaa on mahdollista edelleen nostaa, kun Finnairin omia IT prosesseja saadaan edelleen kehitettyä ja opitaan toimimaan pilvipalveluiden kanssa. Yrityksen historia painottuu pitkältä ajalta konesalipalveluihin ja muutostyö vielä aikaa. Vasta myöhemmin AWS palveluiden arvosana voi nousta.  