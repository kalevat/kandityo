\chapter{Johdanto\label{johdanto}}
Yrityksien keskustietokoneet eli palvelimet ovat perinteisesti sijainneet yrityksen omissa tiloissa tai sijoitettuna isomman yrityksen konesaliin. Viimeiset noin kymmenen vuotta yleisenä mallina on ollut ulkoistaa konesalipalvelut, koska konesalit harvemmin kuuluvat yrityksen ydinliiketoimintaa ja konesalien hallinnointi ympärivuorokautisena palveluna on kallista \citep{data_center_outsourcing}. Viimeisten vuosien aikana suuntana on ollut siirtää palveluita ja palvelimia pois konesaleista pilvipalveluihin. Tällä on haettu kustannussäästöjä sekä palvelun laadun parannusta. Lisäksi osasta palvelimista on luovuttu ja siirrytty käyttämään \emph{serverless} palveluja, joissa pilvipalvelusta ostetaan vain laskentapalvelua. \emph{Serverless} palveluissa ei ole käytössä fyysistä eikä virtuaalipalvelinta \citep{serverless_computing}.

%Tutkimuksen tausta…

Vuoden 2020 keväällä alkanut korona pandemia aiheutti isoja taloudellisia ongelmia kaikille teollisuuden aloille. Tämä vaikutti myös merkittävästi lentoteollisuuteen ja suurin osa lentoliikenteestä jouduttiin keskeyttämään. Korona-aika aiheutti isoja tappioita kaikille lentoyhtiöille. Myös Finnairille lentoliikenteen pysähtyminen aiheutti isoja kustannusongelmia. Finnairilla aloitettiin laajat kustannussäästötoimenpiteet, joilla oli vaikutusta myös IT palveluihin. Olemassa olevia IT palveluja tuli jatkossa ostaa halvemmalla. Finnairilla nähtiin, että tämä hyvänä ajankohtana tehdä konesalipalveluissa seuraava digiloikka. Lentoliikenteessä oli hiljainen ajankohta, joten IT palveluiden muutoksien toteuttamiselle oli pienempi riski. Yrityksen johdon määräys oli, että kustannussäästöjä tuli saavuttaa kaikilla yrityksen liiketoimintasektoreilla, joten konesalipalveluista myös etsittiin säästöjä. Lisäksi yrityksen X kanssa voimassa ollut konesalipalvelusopimus oli päättymässä. Finnairin IT johdossa todettiin, että tämä on sopivana hetkenä siirtyä käyttämään pilvipalveluita laajasti yrityksen IT palveluissa \citep{finnair_use_ibm}.

Tutkimuksen tavoitteena oli kerätä tiedot Finnairilla käytössä olevien pilvipalveluiden laadusta ja verrata niitä aikaisemmin yritykseltä X ostettuihin konesalipalveluihin. Tutkimuksessa tehtiin myös subjektiivinen haastattelu, jossa selvitetään miltä uusien pilvipalveluiden laatu näyttää IT alan työntekijöille. Tutkimuksessa selvitettiin onko pilvipalveluiden laatu parempaa kuin perinteisten konesalipalveluiden laatu. Tästä tehtiin johtopäätelmä tekikö Finnair oikean ratkaisun siirtyessään laajamittaisesti pilvipalveluihin. Tutkielmassa selvitettiin saavutettiinko pilvipalveluiden siirtoprojektin alussa määritellyt tavoitteet. Tutkielman ulkopuolelle jätettiin pilvipalveluiden ja konesalipalveluiden kustannukset, koska nämä tiedot eivät ole julkisia. Tutkimuksen tekemiseen oli Finnairin lupa ja kaikki tutkimuksessa esiteltävät tiedot ovat julkisia.

Tutkimuksessa kerättiin tikettitietoa konesalipalveluiden virhetilanteista ja palvelupyynnöistä. Vastaavat tilastotiedot kerättiin nykyisistä pilvipalveluista. Tutkimuksessa verrattiin näitä eri palvelumallien tilastotietoja ja laadittiin niiden pohjalta analyysi. Tutkimuksessa myös haastateltiin kahta Finnairin IT työntekijää, jotka ovat toimineet konesalipalveluiden ja pilvipalveluiden kanssa. Haastatteluista laadittiin yhteenveto. Tutkimuksessa laadittiin loppuanalyysi, jossa verrattiin konesalipalveluiden ja pilvipalveluiden laatuero tuotannossa.

Tutkimuksen tuloksien mukaan pilvipalveluiden valinta oli strategisesti oikea ratkaisu Finnairille. Ajankohta oli oikea ja muutoksien toteuttamisessa oli pienet riskit. Tutkimuksessa todettiin, että pilvipalveluiden laatu on ollut parempaa. Häiriötilanteista raportoituja tikettejä on ollut lähes yhtä paljon, mutta pilvipalveluissa vakavia häiriötilanteita on ollut vähemmän. Konesalipalveluissa palvelupyynnöt on toteutettu nopeammin käyttäjille. Haastatteluiden perusteella pilvipalveluja pidettiin joustavampana, jolloin muutoksiin ja virhetilanteisiin voidaan reagoida nopeammin. Haastatteluissa arvosanallisesti pilvipalveluja pidettiin parempana vaihtoehtona.
