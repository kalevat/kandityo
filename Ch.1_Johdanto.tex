\chapter{Johdanto\label{johdanto}}
Yrityksien keskustietokoneet eli palvelimet ovat perinteisesti sijainneet yrityksen omissa tiloissa tai sijoitettuna jonkun isommat yrityksen konesaliin. Viimeiset noin kymmenen vuotta yleisenä mallina on ollut ulkoistaa konesalipalvelut, koska se harvemmin kuuluu yrityksen ydinliiketoimintaa ja konesalien hallinnointi 24/7 palveluna on kallista \citep{data_center_outsourcing}. Viimeisten vuosien aikana suunta on ollut siirtää palveluita ja palvelimia pois konesaleista pilvipalveluihin. Tällä on haettu kustannussäästöjä sekä myös palvelun laadun parannusta. Myös osista palvelimista on luovuttu ja on siirrytty käyttämään serverless palveluja, joissa pilvipalvelusta ostetaan vain laskentapalvelua. Serverless palveluissa ei varsinaista fyysisistä tai virtuaalipalvelinta ei enää ole olemassa \citep{serverless_computing}.

Tutkimuksen tausta…

Verrattavissa olevat tutkimukset, menetelmät ja tulokset…


Vuoden 2020 keväällä alkanut korona pandemia aiheutti isoja taloudellisia ongelmia kaikille teollisuuden aloille. Tämä vaikutti myös merkittävästi lentoteollisuuteen ja käytännössä suurin osa lentoliikenteestä keskeytyi. Tämä aiheutti isoja tappioita kaikille lentoyhtiöille. Myös Finnairille lentoliikenteen pysähtyminen aiheuttaa isoja kustannusongelmia. Finnairilla aloitettiin laajat kustannus-sääntötoimenpiteet, joilla oli vaikutusta myös IT palveluihin. Olemassa olevia IT palveluja tuli jatkossa ostaa halvemmalla. Finnairilla nähtiin, että tämä olisi hyvä ajankohta tehdä konesalipalveluissa seuraava digiloikka. Ajankohta oli hiljainen lentoliikenteessä ja ydinliiketoiminnassa, jolloin IT palveluiden muutoksien tekemiselle on pienempi riski. Kustannussäästöjä tuli saavuttaa kaikilla sektoreilla, joten konesalipalveluista niitä myös etsittiin. Lisäksi yrityksen X kanssa voimassa ollut konesalipalvelusopimus oli päättymässä. Finnairin IT johdossa todettiin, että tämä olisi sopiva aika siirtyä käyttämään pilvipalveluita laajasti. \citep{finnair_use_ibm}

Tämä tutkielman tavoitteena on kerätä faktatiedot käytössä olevat pilvipalvelun laadusta vuoden 2021 aikana ja verrata niitä yrityksen X konesalipalveluihin. Tutkielmassa tehdään myös subjektiivinen haastattelu, jossa selvitetään miltä uusien pilvipalveluiden laatu näyttää työntekijöille. Tutkielman tarkoituksena on selvittää, onko pilvipalveluiden laatu parempaa kuin perinteisten konesalipalveluiden laatu. Tästä tehdään johtopäätelmiä tekikö Finnair oikea ratkaisun siirtyessään täysmittaisesti pilvipalveluihin. Tutkielmassa selvitetään saavutettiinko pilvipalveluiden siirtoprojektin alussa määritellyt tavoitteet. Tutkielman ulkopuolelle jätetään pilvipalveluiden ja konesalipalveluiden kustannukset, koska nämä tiedot eivät ole julkisia. Tutkielman laaditaan yhteenveto tutkimuksesta ja tehdään suositusehdotuksia kuinka pilvipalveluiden laatua ja niiden mittaamista voidaan Finnairilla parantaa. Tutkielman tekemiseen on Finnairin lupa ja kaikki tutkielmassa esiteltävät tiedot ovat julkisia.

Oman tutkimuksen menetelmien ja tulosten esittely…

