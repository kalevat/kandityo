\chapter{Tapaustutkimus\label{tapautustutkimus}}
Tutkimuksen kohteena on Finnair Oyj:n konesalipalveluiden ulkoistus pilvipalveluun ja sen laadun vertailu edeltäneeseen ratkaisuun, jossa palvelu tuotettiin perinteisestä konesalista.

Finnair Oyj:ssa IT järjestelmien kehityspolku on edennyt vuosikymmenien aikana omista palvelimista pilvipalveluihin. Aikoinaan Finnairilla oli kellarissa oma konesali, jossa toimivat kaikki yrityksen palvelimet. Niitä ylläpitivät yrityksen omat työntekijät. Myös fyysiset palvelinlaitteet olivat yrityksen omaisuutta. Tämä tapa ylläpitää palvelimia oli kallista, mutta siihen aikaan ei ollut muita vaihtoehtoja. Seuraava kehityskaari oli ulkoistaa palvelimet ja niiden hallinta siihen erikoistuneelle yritykselle. Tässä tutkimuksessa kyseistä konesaliyritystä kutsutaan nimellä X, koska yrityksen nimi on salaista tietoa eikä sitä voida julkistaa tässä tutkimuksessa. Finnairin palvelimet olivat noin kymmenen vuotta yrityksen X:n konesalissa. Toimintamalli oli kustannustehokas, mutta vuosien aikana palvelussa huomattiin ongelmia. Palvelimien käytettävyys ei ollut parhaalla tasolla sekä tuotannon ongelmien korjauksessa oli viivettä. Tämä kaikki näkyi ulospäin laatuongelmina, jopa loppuasiakkaille asti. Lisäksi konesalipalveluiden kilpailutuksesta oli kulunut useampi vuosi eikä enää ollut varmuutta, oliko yrityksen X tarjoamien palveluiden kustannukset oikealla markkinatasolla.

Finnairissa tehtiin strateginen päätös siirtyä perinteisestä konesalipalveluista täysin pilvipalvelujen käyttäjäksi. Siirto tehtiin loppuvuodesta 2020, jonka jälkeen yrityksen kaikki IT palvelut on tuotettu 100 \% pilvipalveluina. Yrityksen X konesalissa olleet palvelimet on kaikki suljettu ja tietoliikenneyhteydet on katkaistu konesaliin. Vuoden 2021 aikana pilvipalveluja on parannettu ja uusia prosesseja luotu, jotta Finnairin työntekijät osaavat toimia oikein pilvipalveluiden kanssa. Muutoksia on tehty paljon erilaisissa toimintamalleissa.

Vuoden 2021 aikana Finnairilla on sisäisesti keskustelu uusien pilvipalveluiden laadusta. Varsinaisia yhteenvetoja tilanteesta ei ole tehty. Laadulliset ongelmat on kirjattu, mutta laatu ei ole virallisesti mitattu. Uusista pilvipalveluista ei ole luotu laatumittareita eikä tuotannon laatua ole verrattu yrityksen X konesalipalveluiden laatuun. Pilvipalveluiden laatuarvioinnit ovat perustuneet subjektiivisiin arvioihin miltä laatu tuntuu eikä faktaperusteisiin tietoihin. 
