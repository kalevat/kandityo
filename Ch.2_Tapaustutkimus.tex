\chapter{Tapaustutkimus\label{tapautustutkimus}}
Tutkielman kohteena on Finnair Oyj:n konesalipalveluiden ulkoistus pilvipalveluun ja sen laadun vertailu edeltäneeseen ratkaisuun, jossa palvelu tuotettiin perinteisestä konesalista.

Finnair Oyj:ssa kehityspolku omista palvelimista pilvipalveluihin on edennyt vuosikymmenien aikana edellä kuvatulla tavalla. Aikoinaan Finnairilla oli kellarissa oma konesali, jossa toimivat kaikki yrityksen palvelimet. Niitä ylläpitivät yrityksen omat työntekijät. Myös fyysiset palvelinlaitteet olivat yrityksen omaisuutta. Tämä tapa ylläpitää palvelimia oli kallista, mutta siihen aikaan ei ollut muita vaihtoehtoja. Seuraava kehityskaari oli ulkoistaa palvelimet ja niiden hallinta siihen erikoistuneelle yritykselle. Tässä tutkielmassa kyseistä yritystä kutsutaan nimellä konesali X, koska yrityksen nimi on salaista tietoa eikä sitä voida julkistaa tässä tutkielmassa. Finnairin palvelimet olivat noin kymmenen vuotta yrityksen X:n konesalissa. Malli oli kustannustehokas, mutta vuosien aikana palvelussa huomattiin ongelmia. Palvelimien käytettävyys ei ollut parhaalla tasolla sekä ongelmien korjauksessa oli myös viivettä. Tämä kaikki näkyi ulospäin laatuongelmina jopa loppuasiakkaille asti. Lisäksi konesalipalveluiden kilpailutuksesta oli kulunut useampi vuosi eikä yrityksen X tarjoamien palveluiden kustannukset olleet enää oikealla tasolla.

Finnairissa tehtiin strateginen päätös siirtyä perinteisestä konesalipalvelusta täysin pilvipalvelujen käyttäjäksi. Siirto tehtiin vuoden 2020 aikana ja sen jälkeen on kaikki IT palvelut tuotettu 100 prosenttisesti pilvipalveluista. Yrityksen X konesalissa olleet palvelimet on kaikki suljettu ja tietoliikenneyhteydet on katkaistu sinne. Vuoden 2021 aikana palveluja on parannettu ja uusia prosesseja luotu, jotta Finnairin työntekijät osaavat oikein toimia pilvipalveluiden kanssa. Muutoksia on tehty paljon erilaisissa toimintamalleissa.

Vuoden 2021 aikana Finnairilla on sisäisesti keskustelu uusien pilvipalveluiden laadusta. Varsinaisia yhteenvetoja tilanteesta ei ole tehty. Laadullisia ongelmia on kirjattu, mutta niitä ei ole mitattu. Uusista pilvipalveluista ei ole luotu laatumittareita eikä laatu ole verrattu yrityksen X konesalipalveluiden laatuun. Pilvipalveluiden laatuarvioinnit ovat perustuneet subjektiivisiin arvioihin miltä laatu tuntuu eikä faktaperusteisiin tietoihin. 
