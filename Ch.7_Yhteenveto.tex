\chapter{Yhteenveto\label{conclusions}}
Tutkimuksen kohteena oli Finnair Oyj:n konesalipalveluiden ulkoistus pilvipalveluun ja sen laadun vertailu edeltäneeseen ratkaisuun, jossa palvelu tuotettiin perinteisenä konesalipalveluna. Finnair Oyj:ssa palvelemien ylläpidon kehityspolku on edennyt vuosikymmenien aikana omista palvelimista vuonna 2020 käyttöön otettuihin pilvipalveluihin. Vuoden 2021 aikana Finnairilla on sisäisesti keskustelu pilvipalveluiden laadusta, mutta varsinaista yhteenvetoa ei ole tehty palvelun laadusta. Tutkimuksen tavoitteena oli verrata konesalipalveluiden ja pilvipalveluiden laatua. Tutkimuksen ulkopuolelle jätettiin palvelumallien kustannuksen vertailu. Tutkimuskysymyksenä oli selvittää tekikö Finnair oikean ratkaisun siirtyessään laajamittaisesti pilvipalveluihin ja onko pilvipalveluiden laatu parempaa kuin perinteiden konesalipalveluiden laatu.

Tutkimuksessa taustatietona selvitettiin erilaisia palvelumalleja, joilla konesalipalveluja tuotetaan asiakkaille. Tutkimuksessa perehdyttiin näiden palvelukuvauksiin sisältöihin ja arviointiin erilaisia malleja, millä palvelimien hallinnointia voidaan tarjota asiakkaille. Tutkimuksessa selvitettiin pilvipalveluiden markkinoita ja verrattiin suurempia palvelutarjoavia. Tutkimuksessa perehdyttiin myös IT palveluiden laadun mittaamiseen ja minkälaisia laatuvaatimuksia asetetaan palvelimien ylläpidolle.

Finnairille konesalipalveluita tuotti aikaisemmin yritys X, jonka virallinen nimeä ei käytetty tutkimuksessa yrityssalaisuuksien takia. Tutkimuksessa kerättiin tilastotiedot konesalipalveluiden virhetilanteiden tiketeistä sekä palvelupyynnöistä, joita oli laadittu konesaliympäristön muutoksista. Näitä tilastotietoja verrattiin vastaaviin tietoihin nykyisestä pilvipalvelusta. Tutkimukseen kuului myös kaksi haastattelua, joissa selvitettiin Finnairin IT asiantuntijan ja IT johtajan subjektiiviset mielipiteet palveluiden laadusta konesalipalveluissa ja pilvipalveluissa.

Tutkimuksesta laadittiin analyysi ja yhteenveto. Tutkimuksessa havaittiin, että pilvipalveluihin siirto on vielä kesken, joten laadusta ei ole vielä kokonaiskuvaa. Tutkimuksen perusteella voidaan todeta, että pilvipalveluiden laatu on ollut parempaa ja vakavia häiriötilanteita on ollut vähemmän. Konesalipalveluissa oli enemmän kriittisiä tikettejä kuin pilvipalveluissa. Tutkimuksessa verrattiin myös palvelupyyntötikettejä. Pilvipalveluissa palvelupyyntöjä on ratkaisu hitaammin ja konesalipalveluissa palvelukokemus oli käyttäjille parempi. Haastatteluissa pilvipalveluja pidettiin vakaampana ja ketterämpänä, mikä mahdollistaa nopeamman reagoinnin muutospyyntöihin. Toisaalta ketteryys vaatii myös asiakkaalta tarkkuutta ja hyviä sisäisiä prosesseja. Tutkimuksen haastatteluissa tietoturvaa pidettiin isompana riskinä pilvipalveluissa, koska kaikki tietoturvaratkaisut eivät ole automatisoituja prosesseja.

Tutkimuksessa todettiin, että pilvipalvelut ovat oikea ratkaisu Finnairille ja yrityksen strategian mukainen suunta kohti ketterämpiä toimintamalleja. Pilvipalveluiden siirto on vielä kesken ja tilastoja tulee tarkastella myöhemmin uudestaan. Pilvipalveluissa laatu on ollut parempaa ja vakavia virhetilanteita on ollut vähemmän. Kokonaisuutena häiriötilanteista tikettejä on raportoitu yhtä paljon molemmissa malleissa. Konesalipalveluissa palvelupyyntöjen ratkaisuajat olivat lyhyempiä ja käyttäjäkokemusta sitä kautta parempi. Haastatteluissa pilvipalveluiden vahvuutena pidettiin joustavuutta ja nopeaa reagointia muutoksiin. Pilvipalvelulle annettiin tutkimuksen haastatteluissa parempi arvosana palvelun laadusta.  
